\section{Introduction} \label{intro}
Crystallization is a crucial process in the production of pharmaceuticals and chemicals, demanding precise control of critical quality attributes like particle size distribution (PSD) and crystal morphology to ensure efficacy \cite{Orehek.2020, Ma.2020, Eren.2023}. Achieving a narrow PSD is particularly important for downstream processing and product bioavailability \cite{Eren.2023, Chen.2011}. Continuous crystallization processes, offering benefits like consistency and improved product control compared to batch methods, are gaining attention \cite{Ma.2020, Eren.2023}. 
The Archimedes Tube Crystallizer (ATC) distinguishes itself through a coiled design and air-segmented flow. These features allow for flexible residence times and a narrow residence time distribution (RTD). By approximating ideal plug flow, the ATC creates the conditions necessary for uniform crystal growth and a narrow PSD \cite{Sonnenschein.2022, Sonnenschein.2021, Sonnenschein.2022b}.

Understanding the intricate hydrodynamics and suspension behavior of particles within the ATC is fundamental, as these phenomena significantly impact crucial aspects like agglomeration, breakage, attrition, and potential crystallizer blockage \cite{Sonnenschein.2022b, Nagy.2020}. Computational Fluid Dynamics (CFD) is an essential tool for probing these complex dynamics and simulating the behavior of liquid segments separated by an air phase.

Previous research on ATC hydrodynamics has applied various computational approaches. For instance, our preliminary study utilized a one-way coupled Lagrangian Particle Tracking (LPT) approach integrated with an FEM-based CFD solver \cite{cryst11121466}. We chose this method as a compromise between accuracy and cost for specific particle sizes and low volume fractions. While it accounts for particle motion under hydrodynamic drag, it neglects reciprocal momentum feedback. However, this approach is less accurate for dense suspensions where particles accumulate locally. This is because the Schiller–Naumann drag correlation is valid only up to certain volume fractions at moderate Reynolds numbers \cite{SchillerNaumann1935}. In general particle-laden flow simulations, coupled CFD–DEM approaches \cite{ligggths, KruggelEmden2008SelectionOA, KRUGGELEMDEN2007157} are popular because they can handle very large numbers of particles efficiently. They rely on unresolved particle–fluid coupling, in which fluid–particle interactions are approximated through correlations. This approximation can limit accuracy in dense suspensions compared to particle-resolved Direct Numerical Simulation (DNS).

For a deeper understanding of particle–fluid interactions within the crystallizer DNS offers significant advantages. DNS is a fully resolved approach that requires computational meshes significantly finer than the particle size, such that hydrodynamic forces and torques emerge directly from the solution of the Navier–Stokes equations. This contrasts with unresolved approaches (e.g. CFD–DEM), where forces are based on correlations and volume fractions, or one-way coupling, where particle feedback is ignored. DNS methods establish full two-way coupling between the fluid and particles. This is achieved through strategies such as the Immersed Boundary Method (IBM) \cite{peskin, uhlmann}, Fictitious Boundary Method (FBM) \cite{WanTurek2006a, fbm2012}, and Fictitious Domain Method (FDM) \cite{glowinski1, patankar2000}. These approaches allow for the accurate resolution of suspension-induced rheology and particle migration phenomena \cite{rod, review2012}.

The high fidelity of DNS makes it a promising tool for obtaining fundamental insights into ATC hydrodynamics and particle behavior. For example, DNS studies in helically coiled tubes have already indicated that particle residence time can be size-dependent due to secondary flow patterns \cite{yang2004}, and more generally, DNS has revealed suspension behaviors that bridge the gap between dilute and dense regimes \cite{PhysRevLett.107.188301, PhysRevLett.109.118305, PhysRevLett.129.078001}.
This level of detail is crucial for understanding complex phenomena that are sensitive to local conditions, such as the interplay between the hydrodynamic environment and particle behavior. Simpler models may fail to fully capture these nuances.

Despite its advantages, DNS comes with significant challenges, primarily related to computational cost. High-fidelity DNS simulations, especially when extended with Population Balance Equations (PBEs), demand substantial computational resources \cite{review2012}. Simulating particle counts above $10^4$ remains highly demanding, which is a major reason why unresolved approaches are still favored for process-scale studies. Nevertheless, the development of efficient multigrid FEM solvers \cite{WanTurek2007a, WanTurek2007b}, as well as advances in parallel DNS frameworks \cite{münster2025effectiveviscosityclosuresdense, münster2025chimeradomaindecompositionmethod}, are gradually expanding the feasible limits of DNS for multiparticle suspensions.


Given this background, the present work aims to explore the applicability and potential of the DNS approach for simulating particle behavior in the ATC. The goals of this work are twofold:
\begin{itemize}
    \item To validate the DNS approach against available experimental data and results from other validated computational methods for representative ATC operating conditions.
    \item To prove the feasibility of DNS for simulating a substantial number of particles relevant to ATC conditions, provide arguments that underline the benefits of DNS in ATC simulations and identifying strategies for handling larger particle counts in future, more comprehensive studies.
\end{itemize}

We present a fully resolved DNS framework that reveals detailed ATC dynamics, establishing a mechanistic foundation for high-fidelity crystallizer design. 

