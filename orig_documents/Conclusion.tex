\section{Conclusion}

This work demonstrated the feasibility of fully resolved DNS for predicting particle suspension states in the Archimedes Tube Crystallizer (ATC). Across eight operating conditions, the simulations reproduced the experimentally established flow map with high fidelity, capturing transitions between green, yellow, red/yellow, and red regimes.  

A central outcome of this study is the introduction of three quantitative descriptors: the axial distribution $\overline{\phi}(s)$, the radial distribution index $I_r$, and the vertical asymmetry $A_y$ (Table~\ref{tab:Ir_Ay_values}). These metrics proved indispensable for identifying distinctive features of each regime, such as rear-loading, radial depletion, or symmetry loss, and for distinguishing subtle transitions that qualitative inspection alone cannot resolve. Importantly, the combined interpretation of all three avoids misleading conclusions that may arise from any single measure in isolation. Beyond validation, the metrics provide a practical diagnostic tool: the corkscrew-like geometry of the ATC makes experimental imaging difficult, while regime classification based on numerical values can be done quickly and unambiguously. The additional insights into particle distribution also have direct relevance for crystallization, where mixing, growth, and agglomeration are strongly affected by suspension quality.  

At the same time, this study strengthens the case for DNS as a method. Unlike one-way coupled or unresolved approaches, the DNS resolves momentum exchange without empirical closures, naturally captures stacking and wall-ward clustering, and remains valid in dense regimes where drag-based models fail. It thereby offers mechanistic transparency and predictive accuracy, enabling not only reliable regime classification but also the development of closures and design strategies grounded in first principles.  

Taken together, the DNS framework and the proposed metrics establish a sound and versatile foundation for ATC suspension analysis. They bridge qualitative flow maps with quantitative measures, reduce experimental ambiguity, and provide mechanistic insight that is essential for advancing crystallizer design. Future work will extend DNS to larger particle numbers, integrate it with population balance models, and further explore how these metrics can inform optimization and scale-up of the ATC concept.
